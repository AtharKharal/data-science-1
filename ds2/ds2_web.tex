\documentclass[compress, black]{beamer}
\setbeamercolor{normal text}{fg=black}
\beamertemplatesolidbackgroundcolor{white}
\usecolortheme[named=black]{structure} 
\usepackage{caption}
\captionsetup{labelformat=empty}
\setbeamertemplate{navigation symbols}{} 
%\usefonttheme{structurebold}
\usepackage[scaled]{helvet}
\renewcommand*\familydefault{\sfdefault} %% Only if the base font of the document is to be sans serif
\usepackage[T1]{fontenc}
\usepackage{setspace}
%\usepackage{beamerthemesplit}
\usepackage{graphics}
\usepackage{hyperref}
\usepackage{graphicx}
\usepackage{verbatim}
\usepackage{amssymb}
\usepackage{wrapfig}
\usefonttheme[onlymath]{serif}
\usepackage{cmbright}

\def\labelitemi{\textemdash}
\setbeamertemplate{frametitle}{
	\begin{centering}
		\vskip15pt
		\insertframetitle
		\par
	\end{centering}
} 
\title[DS]{Get, Clean Data}
\author[Sood]{Gaurav~Sood}
  \large
\date[2015]{Spring 2015}
\subject{LearnDS}
\begin{document}
\newcommand{\multilineR}[1]{\begin{tabular}[b]{@{}r@{}}#1\end{tabular}}
\newcommand{\multilineL}[1]{\begin{tabular}[b]{@{}l@{}}#1\end{tabular}}
\newcommand{\multilineC}[1]{\begin{tabular}[b]{@{}c@{}}#1\end{tabular}}

\newenvironment{large_enum}{
\Large
\begin{itemize}
  \setlength{\itemsep}{7pt}
  \setlength{\parskip}{0pt}
  \setlength{\parsep}{0pt}
}{\end{itemize}}

\begin{comment}

	setwd(paste0(basedir, "github/data-science/ds2/"))
	tools::texi2dvi("ds2_web.tex", pdf=TRUE,clean=TRUE)
	setwd(basedir)

\end{comment}

 \frame
  {
    \titlepage
  }
 
\frame{
	\frametitle{Data, Data, Everywhere}
	 \begin{large_enum}
		\item[--]<2->The Famous Five:\\\normalsize
					  Aural, Visual, Somatic, Gustatory, Olfactory		
		\item[--]<3->The Social Famous Five:\\\normalsize
					  What people (like to) hear, see, sense, smell, taste, \ldots
		\item[--]<4->Manifest Data:\\\normalsize
					  Likes, Ratings, Reviews, Comments, Views, Searches \ldots
		\item[--]<5->Data about data:\\\normalsize
					  Location of a tweet, photo, who called whom, \ldots
		\item[--]<6->Social data:\\\normalsize
					  Friend graph, followers, who retweeted, liked,\ldots
		\item[--]<7->Data about structure:\\\normalsize\vspace{-.4\baselineskip} Layout of the site, In/out links, \ldots 
	\end{large_enum}
}
	
\frame{
	\frametitle{Collecting Digital Data}
	 \begin{large_enum}
		 \item[--]<1->Proprietary Data collections\\\normalsize
		 			 Lexis-Nexis, comScore \ldots
		 \item[--]<2->APIs	\\\normalsize
		 			 Facebook, \href{http://developer.nytimes.com/docs}{NY Times}, Twitter, Google, FourSquare, \href{dfr.jstor.org}{Jstor}, Zillow \ldots
	 	 \item[--]<3->Bulk Downloads \\\normalsize
		 			 Wikipedia, data.gov, IMDB, Million Song Database,  Google n-grams \ldots
		 \item[--]<4->Scraping
		 \item[--]<5->Custom Apps\\\normalsize Build custom apps to observe behavior, get (pay) people to download these apps
	 \end{large_enum}
}

\frame{
	\frametitle{Scraping}
	 \begin{large_enum}
		 \item[--]<1->To analyze data, we typically need structure.\\\normalsize 
		 			  For instance, same number of rows for each column.
		 \item[--]<2->But found data often with human readable structure.
		 \item[--]<2->Copy and paste, type, to a dataset.
		 \item[--]<4->But error prone, and not scalable. 
		 \item[--]<5->\alert{Idea:} Find the less accessible structure, automate based on it.  
	 \end{large_enum}
}

\frame{
	\frametitle{Collecting Found Digital Data}
	\begin{large_enum}
	 \item[-]<1->Software
	 			\begin{enumerate}
	 			\item[-]<2->R - Not the best but will do.
				\item[-]<3->Python, Ruby, Perl, Java, \ldots
				\item[-]<4->30 Digits, 80 Legs, Grepsr \ldots
				\end{enumerate}
	  \item[-]<5->Some things to keep in mind
	  			\begin{enumerate}
	 			\item[-]<6->Check if there is an API, or if data are available for download
	 			\item[-]<7->Play Nice: \\\pause \pause \pause \pause \pause \pause \pause 
	 						- Scraper may be disallowed in `robots.txt' \\ \pause 
	 						- Build lag between requests. \alert{Make lags random.}\\\pause 
	 						- Scrape during off-peak hours
	 			\end{enumerate}
	 \end{large_enum}
	 			
}

\begin{frame}
\frametitle{Paper}
\only<1>{\scalebox{0.35}{\includegraphics{ScannedBook.png}}}
\only<2->{
	\begin{large_enum}
	\item[-]<2-> Create digital images of paper
	\item[-]<3-> Identify colored pixels as characters (OCR) 
	\item[-]<4-> Software
		\begin{enumerate}
		 \item[-]<5->Adobe Pro., etc. 
		 \item[-]<6->Best in class commercial: Abbyy FineReader \\
		 			  Now has an API 
	 	 \item[-]<7->Best in class open-source: Tesseract
		\end{enumerate}
	\item[-]<8->Scrape off recognized characters: pyPdf etc.
	\item[-]<9->Post-processing
	\end{large_enum}
}
\end{frame}

\begin{frame}
\frametitle{Pictures, Audio, and Video}
	\begin{large_enum}
		\item[-]<1->Audio (or Video with audio) to text: Dragon Dictates, Google transcription 
		\item[-]<2->Pictures: recognize color, faces 
		\item[-]<3->Objects in images: \href{clarifai.com}{Clarifai}
		\item[-]<4->Scrape closed-captions
	\end{large_enum}
\end{frame}

\begin{frame}
\frametitle{Get Others to Work}
	\begin{large_enum}
		\item[-]<1->Human Computing
		\item[-]<2->Amazon.com's Mechanical Turk
			\begin{enumerate}
				\item[-]<3-> Create Human Intensive Tasks (HITs)
				\item[-]<4-> \href{https://www.mturk.com/mturk/findhits?match=false}{Surveys, transcription, translation, \ldots}
				\item[-]<5-> You assess the work and pay out 
			\end{enumerate}
		\item[-]<6->Odesk, elance, impact sourcing, run your own ads \ldots
		\item[-]<7->\href{http://www.google.com/insights/consumersurveys/home}{Google} -- surveys as payment for content
\end{large_enum}

\end{frame}


\begin{frame}[fragile]
\frametitle{Scraping one HTML page in Python} 

Shakespeare's Twelfth Night\\
Using \href{http://www.crummy.com/software/BeautifulSoup/}{Beautiful Soup}
\small
	\begin{enumerate}
		\item[]<2->\begin{verbatim} from BeautifulSoup import BeautifulSoup \end{verbatim}
		\item[]<3->\begin{verbatim} from urllib import urlopen \end{verbatim}
		\item[]<3->
		\item[]<4->\begin{verbatim} url  = urlopen(`http://bit.ly/1D7wKcH').read()\end{verbatim}
		\item[]<5->\begin{verbatim} soup = BeautifulSoup(url)\end{verbatim}
		\item[]<6->\begin{verbatim} text = soup.p.contents\end{verbatim}
		\item[]<7->\begin{verbatim} print text\end{verbatim}
	\end{enumerate}
\end{frame}

\begin{frame}[fragile]
\frametitle{Getting text from one pdf in Python} 

A Political Ad\\
Using \href{http://pybrary.net/pyPdf/}{PyPdf}
\small
	\begin{enumerate}
		\item[]<1->\begin{verbatim} import pyPdf \end{verbatim}
		\item[]<2->
		\item[]<2->\begin{verbatim} pdf = pyPdf.PdfFileReader(file('path to pdf', `rb'))\end{verbatim}
		\item[]<3->\begin{verbatim} content = pdf.getPage(0).extractText()\end{verbatim}
		\item[]<4->\begin{verbatim} print content\end{verbatim}
	\end{enumerate}
\end{frame}

\begin{frame}[fragile]
	\frametitle{Scraping many urls/files to structured data}
	\begin{large_enum}
		\item[-]<1->Loop, exploiting structure of the urls/file paths\\\normalsize \pause
					 e.g. \href{http://search.espncricinfo.com/ci/content/match/search.html?search=odi;all=1;page=1}{ESPN URL}
		\item[-]<3->Handle errors, if files or urls don't open, what do you do?
		\item[-]<4->To harvest structured data, exploit structure within text
		\item[-]<5->Trigger words, html tags, \ldots 
	\end{large_enum}
\end{frame}

\begin{frame}[fragile]
	\frametitle{Exception(al) Handling}
	\begin{enumerate}
		\item[]<1->\begin{verbatim}try: \end{verbatim}
		\item[]<1->\begin{verbatim}     pdf = pyPdf.PdfFileReader(file(pdfFile, 'rb')) \end{verbatim}
		\item[]<2->\begin{verbatim}except Exception, e:\end{verbatim}
		\item[]<2->\begin{verbatim}     return `Cannot Open: %s with error: %s' % 
												(pdfFile, str(e))\end{verbatim}
	\end{enumerate}
\end{frame}

\begin{frame}[fragile]
	\frametitle{Inside the page}
	\begin{enumerate}
		\item[-]<1->Chrome Developer Tools
		\item[-]<2->Quick Tour of HTML
			\begin{enumerate}
			\item[-]<3->Tags begin with < and end with >
			\item[-]<4->Tags usually occur in pairs. Some don't (see img). And can be nested.
			\item[-]<5->\href{https://developer.mozilla.org/en-US/docs/Web/HTML/Element}{Mozilla HTML elements}
			\item[-]<6-><p> is for paragraph
			\item[-]<7-><a> is for a link
			\item[-]<8-><ol>, <ul> is for ordered, unordered list; <li> is a bullet
			\item[-]<9->tags can have attributes. <a href='http://somesite'></a>
			\item[-]<10->DOM, hierarchical, parent, child:
						\begin{verbatim} 
									<html>
									    <body>
									        <p></p>
									    </body>
									</html>
				\end{verbatim}
		\end{enumerate}
	\end{enumerate}
\end{frame}

\begin{frame}[fragile]
	\frametitle{Find Things}
	\begin{enumerate}
		\item[]<1->Navigate by HTML tags: \begin{verbatim} soup.title, soup.body, soup.body.contents \end{verbatim}
		\item[]<2->Search HTML tags: 	  \begin{verbatim} soup.find_all('a'), soup.find(id="nav1") \end{verbatim}
		\item[]<3->
		\item[]<3->So to get all the urls in a page:
		\item[]<4->\begin{verbatim} for link in soup.find_all('a'): \end{verbatim}
		\item[]<4->\begin{verbatim}      print(link.get('href'))	 \end{verbatim}						
		\item[]<4->
		\item[]<5->\href{http://www.crummy.com/software/BeautifulSoup/bs4/doc/}{Beautiful Soup Documentation}
	\end{enumerate}
\end{frame}

\frame{
	\frametitle{Data Munging}
	\Large
		``Data scientists, according to interviews and expert estimates, spend from \alert{50 percent to 80 percent of their time mired in the mundane labor of collecting and preparing data}, before it can be explored for useful information.''\\\vspace{5em}

		\small \href{http://www.nytimes.com/2014/08/18/technology/for-big-data-scientists-hurdle-to-insights-is-janitor-work.html}{New York Times: For BigData Scientists, `Janitor Work' Is Key Hurdle to Insights}
}
\frame{
	\frametitle{Data Munging}
	\Large
	``In our experience, the tasks of \alert{exploratory data mining and data cleaning constitute 80\% of the effort} that determines 80\% of the value of the ultimate data.''\\\vspace{5em}

	\small Dasu and Johnson, Exploratory Data Mining and Data Cleaning
}

\frame{
	\frametitle{Regular (or Rational) Expressions}
	\begin{large_enum}
		\item[-]<1-> Formal language for specifying text strings
		\item[-]<2-> Stephen Kleene, `inventor' of regular expressions.
		\item[-]<3-> Henry Spencer, behind the {\tt regex} library.
		\item[-]<4-> Descend from {\it finite automata} theory.
		\item[-]<5-> Matching
	\end{large_enum}
}

\begin{frame}
	\frametitle{The most basic regular expression}
	\begin{large_enum}
		\item[-]<1->String literal
		\item[-]<2->\href{http://regexpal.com}{RegexPal.com}
		\item[-]<2->Say you are searching for the word apple -- can be uppercase first character, plural, lowercase first character
	\end{large_enum}
\end{frame}

\begin{frame}[fragile]
	\frametitle{Disjunction}
	\begin{large_enum}
		\item[-]<1->Disjunction, Character classes
				\begin{enumerate}
				\item[-]<2-> \begin{verbatim} [] \end{verbatim}
				\item[-]<3-> \begin{verbatim} [aA]pple matches apple and Apple \end{verbatim}
				\item[-]<4-> \begin{verbatim} [0123456789]  matches any digit \end{verbatim}
				\end{enumerate}
		\item[-]<5->Ranges
			\begin{enumerate}
				\item[-]<6-> \begin{verbatim} [0-9]  matches any digit \end{verbatim}
				\item[-]<7-> \begin{verbatim} [a-z], [[:lower:]]  matches any lowercase \end{verbatim}
				\item[-]<8-> \begin{verbatim} [a-zA-Z], [[:alpha:]]  matches any uppercase \end{verbatim}
				\item[-]<9-> \begin{verbatim} [a-e1-9]  matches any letter or digit \end{verbatim}
				\item[-]<10-> Hyphen only has a special meaning if used within range. \begin{verbatim} [-123] \end{verbatim}
			\end{enumerate}
	\end{large_enum}
\end{frame}

\begin{frame}[fragile]
	\frametitle{Disjunction Contd..}
	\begin{large_enum}
	\item[-]<1->Negation in Disjunction\\\normalsize
			\begin{enumerate}
			\item[-]<2-> \begin{verbatim} ^ right after the square bracket means a negation \end{verbatim}
			\item[-]<3-> \begin{verbatim}  [^A-Z] \end{verbatim}
			\item[-]<4-> \begin{verbatim}  [^Aa] means neither a capital A nor a lowercase a \end{verbatim}
			\item[-]<5-> \begin{verbatim}  [^e^] means not an e, and not ^ \end{verbatim}
			\end{enumerate}
	\item[-]<6->Disjunction for longer strings\\\normalsize
			\begin{enumerate}
			    \item[-]<7-> \begin{verbatim}  pipe \end{verbatim}
				\item[-]<8-> \begin{verbatim}  a|b|c = [abc] \end{verbatim}
				\item[-]<9-> \begin{verbatim}  apple|pie \end{verbatim}
				\item[-]<10-> \begin{verbatim}  [aA]pple|[aA]nd \end{verbatim}
			\end{enumerate}
	\end{large_enum}
\end{frame}
\begin{frame}[fragile]
	\frametitle{Special characters}
	\begin{large_enum}
	\item[-]<1->? - previous character is optional: colou?r - color, colour
	\item[-]<2-> . matches any character\\
				  e.g. beg.n matches begun, begin, began
	\item[-]<3->Kleene Operators - named after Steven Kleene
			\begin{enumerate}
			\item[-]<4->*  matches 0 or more of the previous characters\\
						e.g. oo*h will match ooh, oooh, etc.\\
						(abc)* will match abc, abcabc, etc.
			\item[-]<5->+  matches 1 or more of the previous characters\\
						e.g. o+h will match ooh, oooh, etc. 
			\end{enumerate}
	\end{large_enum}
\end{frame}

\begin{frame}[fragile]
	\frametitle{Repetition Ranges}
	\begin{enumerate}
	\item[-]<1-> Specific ranges can also be specified
	\item[-]<2-> \small \begin{verbatim} { } to specify range for the immediately preceding regex \end{verbatim}
	\item[-]<3-> \begin{verbatim} {n}  means exactly n occurrences \end{verbatim}
	\item[-]<4-> \begin{verbatim} {n,} means at least n occurrences \end{verbatim}
	\item[-]<5-> \begin{verbatim} {n,m} means at least n and no more than m occurrences \end{verbatim}
	\item[-]<6-> Example: \begin{verbatim} . {0, } = .* \end{verbatim}	
	\end{enumerate}
\end{frame}

\begin{frame}[fragile]
	\begin{large_enum}
	\frametitle{More Regex}
	\item[-]<1->Anchors
			\begin{enumerate}
			\item[-]<2-> $^$ matches the beginning of the line\\
	 			  e.g. \begin{verbatim} ^[A-Z] matches a captial letter at the start of a line.\end{verbatim}
			\item[-]<3-> \$ matches the end of the line.
			\end{enumerate}
	\item[-]<4-> \begin{verbatim} \. means a period \end{verbatim} 
	\item[-]<5-> Example: look for the word `the'
	\begin{enumerate}
		\item[-]<6->missed capitalization: [tT]he
		\item[-]<7->make pattern more precise: \\
					\begin{verbatim} [tT]he[^A-Za-z], ^[tT]he[^A-Za-z] \end{verbatim}
	\end{enumerate}
	\end{large_enum}
\end{frame}

\frame{
\frametitle{False Positive and Negatives}
	\begin{large_enum}
	\item[-]<1->False positives or Type 1 errors - matching things we shouldn't match
	\item[-]<2->False negatives or Type 2 errors - not matching things we should match
	\item[-]<3->Cost attached to false negative and positive
	\item[-]<4->Provide some metrics by comparing against good data for a small sample
	\end{large_enum}
}

\frame{
	\frametitle{Edit Distance}
	\begin{large_enum}
		\item[-]<1->pwned -> owned or pawned?
		\item[-]<2->standd -> strand, stand, stood, or sand?
		\item[-]<3->How similar are two strings?
		\item[-]<4->Applications
			\begin{enumerate}
			\item[-]<5->Spell Correction
			\item[-]<6->Also comes up in computational biology
			\item[-]<7->Machine translation
			\item[-]<8->Information extraction
			\item[-]<9->Speech recognition
			\end{enumerate}
	\end{large_enum}
}
\frame{
	\frametitle{Edit Distance}
	\begin{large_enum}
		\item[-]<1->Typically refers to minimum edit distance
		\item[-]<2->Minimum number of editing operations to convert one string to another
			\begin{enumerate}
				\item[-]<3->Insertion
				\item[-]<3->Deletion
				\item[-]<3->Substitution
			\end{enumerate}
		\item[-]<3->e.g. two strings: intention, execution 
			\begin{enumerate}
				\item[-]<4->align it with second letter
				\item[-]<5->d (delete), s (substitute), s, i(nsert), s
				\item[-]<6->if each operation costs 1, edit distance = 5
				\item[-]<7->if substitition cost 2 (levenshtein distance), distance = 8
			\end{enumerate}
		\item[-]<8->You can implement this at word level so Microsoft Corp. is 1 away from Microsoft.
	\end{large_enum}
}

\frame{
	
	\center \Huge Text Processing
}

\frame{
	\frametitle{Text as Data}
	\begin{large_enum}
		\item[-]<1->Bag of words assumption\\\normalsize
					Lose word order
		\item[-]<2->Remove stop words:\\\normalsize
					If, and, but, who, what, the, they, their, a, or, \ldots\\
					\alert{Be careful: one person's stopword is another's key term.}
		\item[-]<3->(Same) Word: Stemming and Lemmatization\\\normalsize
					Taxing, taxes, taxation, taxable $\leadsto$ tax
		\item[-]<4->Remove rare words\\\normalsize
					$\sim$ .5\% to 15\%, depending on application\\
		\item[-]<5->Convert to lowercase, drop numbers, punctuation, etc.
	\end{large_enum}
}

\begin{frame}[fragile]
	\frametitle{How?}
	Using Natural Language Toolkit (\tt{nltk})
	\begin{itemize}
			\item[-]<2->\textbf{Lowercase}: 
 					 	\begin{verbatim}text = text.lower() \end{verbatim} 
 			\item[-]<3->\textbf{Remove stop words}:
	 			\item[]<4->\begin{verbatim}swords = stopwords.words('english') \end{verbatim}
	 			\item[]<5->\begin{verbatim}words  = wordpunct_tokenize(text) \end{verbatim}
	 			\item[]<6->\begin{verbatim}words  = [w for w in words if w not in swords] \end{verbatim}
	    		\item[]<7->\begin{verbatim}text = ' '.join(words) \end{verbatim}
 			\item[-]<8->\textbf{Stemming}: 
 				\item[]<9->\begin{verbatim}st = EnglishStemmer() \end{verbatim}
 				\item[]<10->\begin{verbatim}words = wordpunct_tokenize(text) \end{verbatim}
    			\item[]<11->\begin{verbatim}words = [st.stem(w) for w in words] \end{verbatim}
    			\item[]<12->\begin{verbatim}text = ' '.join(words) \end{verbatim}
		\end{itemize}
\end{frame}
\begin{frame}[fragile]
	\frametitle{To Matrices}
	\begin{itemize}
			\item[-]<2->n-grams
					\begin{verbatim}
						from nltk import bigrams, trigrams, ngrams
						text = word tokenize(text)
						text_bi  = bigrams(text)
					\end{verbatim}
	\end{itemize}
\end{frame}

\end{document}
